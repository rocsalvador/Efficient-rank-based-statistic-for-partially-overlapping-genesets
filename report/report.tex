\documentclass[12pt]{extarticle}
\usepackage[english]{babel}
\usepackage{subfigure}
\usepackage{tikz}
\usetikzlibrary{arrows}
\usepackage{multirow}
\usepackage{float}
\usepackage{listings}
\usepackage{parskip}
\usepackage[bottom]{footmisc}
\usepackage{caption}
\usepackage[utf8]{inputenc}
\usepackage{csquotes}
\usepackage[colorlinks=true,linkcolor=blue,citecolor=red, urlcolor=red]{hyperref}

\begin{document}

\author{Roc Salvador \\ Supervised by Pål Sætrom}

\title{\bfseries Efficient rank-based statistic for partially overlapping gene sets}

\maketitle


\vspace{10cm}

\begin{figure}[H]
    \centering
    \includegraphics[width=5cm]{img/ntnu.png}
\end{figure}

\begin{center}

Department of Computer Science

\end{center}


\newpage

\tableofcontents

\newpage


\section{Abstract}

\paragraph{} Gene set enrichment analyses use rank-based statistics to test whether related genes are uniformly distributed in an ordered gene list. By testing several gene sets with defined biological functions on gene lists from a biological experiment, one can determine whether specific biological functions are significantly affected by the experiment. The goal of this thesis is to develop efficient algorithms for computing rank-based statistics for thousands of gene sets on tens of thousands of gene lists and use these as a basis for analyzing data from single-cell sequencing experiments.

\newpage

\section{Introduction}

\newpage

\section{Methods}

\newpage

\section{Results}

\end{document}
